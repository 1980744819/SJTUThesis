%# -*- coding: utf-8-unix -*-
%%==================================================
%% thesis.tex
%%==================================================

% 双面打印
\documentclass[degree=doctor,fontset=fandol, openright, twoside]{sjtuthesis}
% \documentclass[bachelor, openany, oneside, submit]{sjtuthesis}
% \documentclass[master, review]{sjtuthesis}
% \documentclass[%
%   bachelor|master|doctor, % 必选项
%   fontset=fandol|windows|mac|ubuntu|adobe|founder, % 字体选项
%   oneside|twoside,        % 单面打印,双面打印(奇偶页交换页边距,默认)
%   openany|openright,      % 可以在奇数或者偶数页开新章|只在奇数页开新章(默认)
%   english,                % 启用英文模版
%   review,     % 盲审论文,隐去作者姓名、学号、导师姓名、致谢、发表论文和参与的项目
%   submit      % 定稿提交的论文,插入签名扫描版的原创性声明、授权声明 
% ]
\usepackage{sjtuextra}

\sjtusetup{
  titleZh={上海交通大学学位论文 \LaTeX 模板示例文档},
  titleEn={A Sample Document for \LaTeX-basedd SJTU Thesis Template},
  authorZh={某\quad{}某},
  authorEn={\textsc{Mo Mo}},
  id={0010900990},
  advisorZh={某某教授},
  advisorEn={Prof. \textsc{Mou Mou}},
  % coadvisorZh={某某教授},
  % coadvisorEn={Prof. \textsc{Mou Mou}},
  instituteZh={某某系},
  instituteEn={\textsc{Depart of XXX, School of XXX} \\
    \textsc{Shanghai Jiao Tong University} \\
    \textsc{Shanghai, P.R.China}},
  majorZh={某某专业},
  majorEn={A Very Important Major},
  dateZh={2014年12月17日},
  dateEn={Dec. 17th, 2014},
}

%% Graphic path & file extension
\graphicspath{{fig/}{figure/}{figures/}{logo/}{logos/}{graph/}{graphs}}
\DeclareGraphicsExtensions{.pdf,.eps,.png,.jpg,.jpeg}

% 逐个导入参考文献数据库
\addbibresource{bib/thesis.bib}
% \addbibresource{bib/chap2.bib}

\begin{document}

% 无编号内容:中英文论文封面、授权页
%# -*- coding: utf-8-unix -*-
% !TEX program = xelatex
% !TEX root = ../thesis.tex
% !TEX encoding = UTF-8 Unicode
%TC:ignore
\title{基于深度强化学习的人机控制}
\author{资霄}
\advisor{李辉}
% \coadvisor{某某教授}
\defenddate{2019年5月17日}
\coursename{某某课程}
\school{北京化工大学}
\institute{信息科学与技术}
\studentnumber{2015014325}
\cnacademicdegree{工学硕士}
\major{计算机科学与技术}
\keywords{深度学习,强化学习,AI,DQN,游戏}

\englishtitle{A Sample Document for \LaTeX-based SJTU Thesis Template}
\englishauthor{\textsc{Mo Mo}}
\englishadvisor{Prof. \textsc{Mou Mou}}
% \englishcoadvisor{Prof. \textsc{Uom Uom}}
\englishschool{Shanghai Jiao Tong University}
\englishinstitute{\textsc{Depart of XXX, School of XXX} \\
  \textsc{Shanghai Jiao Tong University} \\
  \textsc{Shanghai, P.R.China}}
\englishinstitutemaster{Depart of XXX, \\ School of XXX}
\englishmajor{A Very Important Major}
\englishdate{Dec. 17th, 2014}
\enacademicdegree{Master of Engineering}
\englishstudentid{0010900990}
\englishkeywords{SJTU, master thesis, XeTeX/LaTeX template}
%TC:endignore

\maketitle

\makeatletter
\ifsjtu@submit\relax
  \includepdf{pdf/original.pdf}
  \cleardoublepage
  \includepdf{pdf/authorization.pdf}
  \cleardoublepage
\else
\ifsjtu@review\relax
% exclude the original claim and authorization
\else
  \makeDeclareOriginal
  \makeDeclareAuthorization
\fi
\fi
\makeatother

\frontmatter % 使用罗马数字对前言编号

% 摘要
%# -*- coding: utf-8-unix -*-
% !TEX program = xelatex
% !TEX root = ../thesis.tex
% !TEX encoding = UTF-8 Unicode
%%==================================================
%% abstract.tex for SJTU Master Thesis
%%==================================================

\begin{abstract}

    自从DeepMind团队在2013年提出DQN学习算法以及之后使用AlphaGo打败李世石,深度强化学习名声大噪.很多人把深度强化学习运用到不同的游戏控制,机器人控制,导航等领域,并取得了超越传统强化学习的显著效果。在本篇文章中,笔者将实现将DQN算法以及DQN算法的三大改进结合起来,并将其应用到1985NES一款游戏《超级玛丽》上,控制游戏人物在不死亡的情况下运动到距离游戏起点尽可能远的位置,直到通过第一关。本课题成功控制游戏人物在游戏中已经能达到接近甚至有很大几率达到通过第一关的能力。本课题有着很大的应用前景,可以将其成果应用到游戏人机对战,游戏AI控制,机器人自主导航,甚至无人驾驶等各种领域。

\end{abstract}




% 目录、插图目录、表格目录
\tableofcontents
\listoffigures
\addcontentsline{toc}{chapter}{\listfigurename}     % 将插图目录加入全文目录
\listoftables
\addcontentsline{toc}{chapter}{\listtablename}      % 将表格目录加入全文目录
\listofalgorithms
\addcontentsline{toc}{chapter}{\listalgorithmname}  % 将算法目录加入全文目录

\include{tex/symbol} % 主要符号、缩略词对照表

\mainmatter % 使用阿拉伯数字对正文编号

% 正文内容
% \include{tex/intro}
\include{tex/example}
\include{tex/faq}
\chapter{总结}
本文介绍了深度强化学习的发展与目前的研究内容,重点介绍了DQN算法及其三大改进:double DQN,Prioritized Experience Replay DQN,Prioritized,Dueling DQN。其中double DQN是关于DQN公式(更新方法)的改进,Prioritized Experience Replay是关于经验回放的改进,Dueling DQN是关于神经网络结构的改进。
在本课题中,笔者成功将DQN算法及其三大改进算法结合起来应用到超级玛丽这款经典游戏的控制上,取得了明显的学习效果。
在取得了明显的学习效果中,还存在着许多不足之处,如收敛不够,神经网络分类不准确,对于长期的价值掌握不够等。还需要在这些方面做进一步改进。通过本次课题对深度强化学习有了充分的认知理解,掌握了DQN算法背后所蕴含的思想,深度学习的发展带领人工智能向前迈进了一大步。深度强化学习无论是在理论上还是在实际应用中都有着巨大的前景,目前深度强化学习还有很广阔的发展空间。

\appendix % 使用英文字母对附录编号

% 附录内容,本科学位论文可以用翻译的文献替代。
%# -*- coding: utf-8-unix -*-
% !TEX program = xelatex
% !TEX root = ../thesis.tex
% !TEX encoding = UTF-8 Unicode
\chapter{搭建模板编译环境}

\section{安装TeX发行版}

\subsection{Mac OS X}

Mac用户可以从MacTeX主页\footnote{\url{https://tug.org/mactex/}}下载MacTeX。
也可以通过brew包管理器\footnote{\url{http://caskroom.io}}安装MacTeX。

\begin{lstlisting}[basicstyle=\small\ttfamily, numbers=none]
brew cask install mactex
\end{lstlisting}

\subsection{Linux}

建议Linux用户使用TeXLive主页\footnote{\url{https://www.tug.org/texlive/}}的脚本来安装TeXLive。
以下命令将把TeXLive发行版安装到当前用户的家目录下。
若计划安装一个供系统上所有用户使用的TeXLive,请使用root账户操作。

\begin{lstlisting}[basicstyle=\small\ttfamily, numbers=none]
wget http://mirror.ctan.org/systems/texlive/tlnet/install-tl-unx.tar.gz
tar xzvpf install-tl-unx.tar.gz
cd install-tl-20150411/
./install-tl
\end{lstlisting}

\section{安装中文字体}

\subsection{Mac OS X、Deepin}

Mac和Deepin用户双击字体文件即可安装字体。

\subsection{RedHat/CentOS用户}

RedHat/CentOS用户请先将字体文件复制到字体目录下,调用fc-cache刷新缓存后即可在TeXLive中使用新字体。

\begin{lstlisting}[basicstyle=\small\ttfamily, numbers=none]
mkdir ~/.fonts
cp *.ttf ~/.fonts				# 当前用户可用新字体
cp *.ttf /usr/share/fonts/local/	# 所有用户可以使用新字体
fc-cache -f
\end{lstlisting}


\include{tex/app_eq}
% \include{tex/app_cjk}
%# -*- coding: utf-8-unix -*-
% !TEX program = xelatex
% !TEX root = ../thesis.tex
% !TEX encoding = UTF-8 Unicode
\chapter{模板更新记录}
\label{chap:updatelog}

\textbf{2018年1月} v0.10发布,项目转移至 \href{https://github.com/sjtug/SJTUThesis}{SJTUG} 名下,并增加了英文模版,修改了默认字体设置。

\textbf{2016年12月} v0.9.5发布,改用GB7714-2015参考文献风格。

\textbf{2016年11月} v0.9.4发布,增加算法和流程图。

\textbf{2015年6月19日} v0.9发布,适配ctex 2.x宏包,需要使用TeXLive 2015编译。

\textbf{2015年3月15日} v0.8发布,使用biber/biblatex组合替代 \BibTeX ,带来更强大稳定的参考文献处理能力;添加enumitem宏包增强列表环境控制能力;完善宏包文字描述。

\textbf{2015年2月15日} v0.7发布,增加盲审选项,调用外部工具插入扫描件。

\textbf{2015年2月14日} v0.6.5发布,修正一些小问题,缩减git仓库体积,仓库由sjtu-thesis-template-latex更名为SJTUThesis。

\textbf{2014年12月17日} v0.6发布,学士、硕士、博士学位论文模板合并在了一起。

\textbf{2013年5月26日} v0.5.3发布,更正subsubsection格式错误,这个错误导致如"1.1 小结"这样的标题没有被正确加粗。

\textbf{2012年12月27日} v0.5.2发布,更正拼写错误。在diss.tex加入ack.tex。

\textbf{2012年12月21日} v0.5.1发布,在 \LaTeX 命令和中文字符之间留了空格,在Makefile中增加release功能。

\textbf{2012年12月5日} v0.5发布,修改说明文件的措辞,更正Makefile文件,使用metalog宏包替换xltxtra宏包,使用mathtools宏包替换amsmath宏包,移除了所有CJKtilde(\verb+~+)符号。

\textbf{2012年5月30日} v0.4发布,包含交大学士、硕士、博士学位论文模板。模板在\href{https://github.com/sjtug/SJTUThesis}{github}上管理和更新。

\textbf{2010年12月5日} v0.3a发布,移植到 \XeTeX/\LaTeX 上。

\textbf{2009年12月25日} v0.2a发布,模板由CASthesis改名为sjtumaster。在diss.tex中可以方便地改变正文字号、切换但双面打印。增加了不编号的一章“全文总结”。
添加了可伸缩符号(等号、箭头)的例子,增加了长标题换行的例子。

\textbf{2009年11月20日} v0.1c发布,增加了Linux下使用ctex宏包的注意事项、.bib条目的规范要求,
修正了ctexbook与listings共同使用时的断页错误。

\textbf{2009年11月13日} v0.1b发布,完善了模板使用说明,增加了定理环境、并列子图、三线表格的例子。

\textbf{2009年11月12日} 上海交通大学硕士学位论文 \LaTeX 模板发布,版本0.1a。



\backmatter % 文后无编号部分 

% 参考资料
\printbibliography[heading=bibintoc]

% 致谢、发表论文、申请专利、参与项目、简历
% 用于盲审的论文需隐去致谢、发表论文、申请专利、参与的项目
\makeatletter

% "研究生学位论文送盲审印刷格式的统一要求"
% http://www.gs.sjtu.edu.cn/inform/3/2015/20151120_123928_738.htm

% 盲审删去删去致谢页
\ifsjtu@review\relax\else
  %# -*- coding: utf-8-unix -*-
% !TEX program = xelatex
% !TEX root = ../thesis.tex
% !TEX encoding = UTF-8 Unicode
\begin{thanks}

  感谢所有测试和使用交大学位论文 \LaTeX 模板的同学!

  感谢那位最先制作出博士学位论文 \LaTeX 模板的交大物理系同学!

  感谢William Wang同学对模板移植做出的巨大贡献!
  
  感谢 \href{https://github.com/weijianwen}{@weijianwen} 学长一直以来的开发和维护工作!
  
  感谢 \href{https://github.com/sjtug}{@sjtug} 以及 \href{https://github.com/dyweb}{@dyweb} 对 0.9.5 之后版本的开发和维护工作!
  
  感谢所有为模板贡献过代码的同学们, 以及所有测试和使用模板的各位同学!

\end{thanks}
         % 致谢
\fi

\ifsjtu@bachelor
  % 学士学位论文要求在最后有一个英文大摘要,单独编页码
  \include{tex/end_english_abstract}
\else
  % 盲审论文中,发表学术论文及参与科研情况等仅以第几作者注明即可,不要出现作者或他人姓名
  \ifsjtu@review\relax
    \include{tex/pubreview}
    \include{tex/projectsreview}  
  \else
    \include{tex/pub}       % 发表论文
    \include{tex/projects}  % 参与的项目
  \fi
\fi

% \include{tex/patents}     % 申请专利
% \include{tex/resume}      % 个人简历

\makeatother

\end{document}
